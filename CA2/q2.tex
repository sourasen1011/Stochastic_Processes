\textbf{2}. Consider the following Markov chain with transition matrix:

\[
\begin{pmatrix}
  0 & 1 & 0 & 0 & 0 \\
  1-\alpha & 0 & \alpha & 0 & 0 \\
  0 & 1 & 0 & 0 & 0 \\
  \beta & 0 & 0 & 0 & 1-\beta \\
  0 & 0 & 0 & \frac{1}{2} & \frac{1}{2}
\end{pmatrix}
\]

Here $T_{ij} = P(X_n = j | X_{n-1} = i)$ for all $i,j \in \{1,2,3,4,5\}$, and $\alpha$ and $\beta$ do not depend on $n$. 

(a) The state transition diagram is given below
\begin{figure}[H]
\centering
\includegraphics[scale=0.5]{T-ij.jpg}
\caption{\label{fig:T-ij}State transition diagram\ \textbf{Answer}}
\end{figure}
\textbf{State 2}
To prove that state 2 is positively recurrent, we need to prove that the mean recurrence time $\mu=\sum_{n=1}^{\infty}nf_{i}^{(n)} < \infty$. First, we check the times of first return.
\begin{gather*}
f_2^{(1)} = 0 \\
f_2^{(2)} = 1 - \alpha + \alpha = 1\\
f_2^{(3)} = 0 \\
f_2^{(4)} = 0 \\
f_2^{(5)} = 0 \\
\vdots 
\end{gather*}
Note that the process cannot return to state 2 for the \textit{first} time after more than 2 steps - by the 3rd or 4th time (and indeed any step more than that), it will already be \textit{revisiting} state 2. Calculate the mean recurrence time $\mu= 1\cdot f_{2}^{(1)} + 2\cdot f_{2}^{(2)} + 3\cdot f_{2}^{(3)} + 4\cdot f_{2}^{(4)} + \cdots$\\
$\implies \mu = 1\cdot 0 + 2\cdot 1 + 3\cdot 0 + 4\cdot 0 + \cdots$\\
$\implies \mu = 2 < \infty$,
Therefore, state 2 is positively recurrent.\ \textbf{Answer}

\textbf{State 5} To see if state 5 is transient, we check if sum of times of first return is $< 1$.
\begin{gather*}
f_5^{(1)} = \frac{1}{2} \\
f_5^{(2)} = \frac{1}{2}(1 - \beta) \\
f_5^{(3)} = 0 \\
f_5^{(4)} = 0 \\
f_5^{(5)} = 0 \\
\vdots\\
\sum_{n=1}^{\infty}f_{5}^{(n)} = \frac{1}{2}(1 - \beta)
\end{gather*}
It is not dependent on $\alpha$, but for $0<\beta<1$, the sum is lesser than 1, which means that state 5 is transient.\ \textbf{Answer}\\
We notice that similar to state 2, the process cannot return to state 5 for the \textit{first} time after more than 2 steps - it will have already visited it before (on the first or second step).

(b) For fixed $\alpha , \beta > 0$, it can be seen that there are two sub-chains: 1-2-3 and 4-5.

Following are the classifications of the states: \ \textbf{Answer}

\textbf{State 1}: It is periodic as for $k = 2$, $(T^n)_{11} > 0$ for all $n = k,2k,3k...$ and $(T^n)_{11} = 0$ for all $n \neq k,2k,3k...$. To elaborate, starting at state 1, the system can return to state 1 after 2,4,6... steps, but cannot return to it after 1,3,5... steps.

\textbf{State 2}: Following the above logic, it can be seen that state 2 is recurrent. The difference from state 1 is that, for state 1, the system can return to it after any even step size (2,4,6,8....), whereas, for state 2, the system must return to it every 2 steps. Mean recurrence time for state 2 is therefore $\mu=2$, as seen earlier in part (a).

\textbf{State 3}: Following exactly the same logic as state 1, it can be seen that state 3 is also periodic with the system returning to it after any even-numbered time step(s).

\textbf{States 4 \& 5}: There is a probability of $\beta$ of the system leaving state 4 (and thus the subchain) and never being able to return to it. So states  4 and 5 area transient.

The criteria for a subchain to be \textit{ergodic} is for it to be both aperiodic and positively recurrent. Suchain 1-2-3 violates the condition for aperiodicity. Suchain 4-5 violates the condition of positively recurrent. Thus, neither subchain is ergodic. \ \textbf{Answer}

(c) To find the steady state probability vector in terms of $\alpha$ and $\beta$., we can solve the eigenvector equation $\tilde{P} = \tilde{P}T$.

\begin{gather*}
\implies [P_1 , P_2 , P_3 , P_4 , P_5] = [P_1 , P_2 , P_3 , P_4 , P_5] \begin{pmatrix}
  0 & 1 & 0 & 0 & 0 \\
  1-\alpha & 0 & \alpha & 0 & 0 \\
  0 & 1 & 0 & 0 & 0 \\
  \beta & 0 & 0 & 0 & 1-\beta \\
  0 & 0 & 0 & \frac{1}{2} & \frac{1}{2}
\end{pmatrix}
\end{gather*}

From the above equation, the probabilities can be arranged as

\begin{equation}\label{eqn:p1}
P_1 = P_2(1-\alpha) + P_4\beta
\end{equation}

\begin{equation}\label{eqn:p2}
P_2 = P_1 + P_3
\end{equation}

\begin{equation}\label{eqn:p3}
P_3 = P_2\alpha
\end{equation}

\begin{equation}\label{eqn:p4}
P_4 = \frac{P_5}{2}
\end{equation}

\begin{equation}\label{eqn:p5}
P_5 = P_4(1-\beta)+\frac{P_5}{2}
\end{equation}

Equation \ref{eqn:p5} can be rewritten as $P_5 = \frac{P_5}{2}(1-\beta) + \frac{P_5}{2}$, solving which we get $P_5=0$ for $\beta>0$. It follows from \ref{eqn:p4} that $P_4=\frac{P_5}{2} = 0$. This also reduces \ref{eqn:p1} to $P_1 = P_2(1-\alpha)$.

We can now say that $P_1 + P_2 + P_3=1$ as the sum of the steady state probabilities must be 1. Solving,
\begin{gather*}
P_1 + P_2 + P_3 = 1\\
P_2(1-\alpha) + P_2 + P_2\alpha = 1\\
P_2(1 - \alpha + 1 - \alpha) = 1\\
P_2 = \frac{1}{2}\ \textbf{Answer}\\
\implies P_1 = \frac{1-\alpha}{2} \ \& \ P_3 = \frac{\alpha}{2}\ \textbf{Answer}
\end{gather*}
(d)  Suppose now $\alpha := \alpha_n = \frac{n}{n + 1}$, so that now $T \equiv T(n)$ is a time dependent transition
matrix. This amends the state transition diagram to 
\begin{figure}[H]
\centering
\includegraphics[scale=0.5]{Tn-ij.jpg}
\caption{\label{fig:Tn-ij}State transition diagram for time-dependent Markov chain}
\end{figure}
To find the classification of state 1, we need to compute $\sum_{n=1}^{\infty}f_{1}^{(n)}$ and $\mu=\sum_{n=1}^{\infty}nf_{1}^{(n)} < \infty$.

The times of first return are
\begin{gather*}
f_1^{(1)} = 0 \\
f_1^{(2)} = \frac{1}{n_2+1} = \frac{1}{3} \\
f_1^{(3)} = 0 \\
f_1^{(4)} = \frac{n_2}{n_2+1}  \frac{1}{n_4+1} = \frac{2}{3}\cdot\frac{1}{5}  \\
f_1^{(5)} = 0\\
f_1^{(6)} = \frac{n_2}{n_2+1}  \frac{n_4}{n_4+1}  \frac{n_6}{n_6+1} = \frac{2}{3}\cdot\frac{4}{5}\cdot\frac{1}{7}  \\
f_1^{(7)} = 0 \\
\vdots\\
\end{gather*}
Unsurpisingly, the probability of every odd-step return is 0.
%+ (\frac{2}{3}\cdot\frac{4}{5}\cdot\frac{6}{7}\cdot\frac{8}{9}\cdot\frac{1}{11}) 
\begin{gather*}
\sum_{n=1}^{\infty}f_{5}^{(n)} = \frac{1}{3} + (\frac{2}{3}\cdot\frac{1}{5}) + (\frac{2}{3}\cdot\frac{4}{5}\cdot\frac{1}{7}) + (\frac{2}{3}\cdot\frac{4}{5}\cdot\frac{6}{7}\cdot\frac{1}{9}) + \cdots\\
\sum_{n=1}^{\infty}f_{5}^{(n)} = \frac{1}{3} + (\frac{2}{3}\cdot(1-\frac{4}{5})) + (\frac{2}{3}\cdot\frac{4}{5}\cdot(1-\frac{6}{7})) + (\frac{2}{3}\cdot\frac{4}{5}\cdot\frac{6}{7}\cdot(1-\frac{8}{9})) + \cdots\\
\sum_{n=1}^{\infty}f_{5}^{(n)} = \frac{1}{3} + (\frac{2}{3} - \frac{2}{3}\cdot\frac{4}{5}) + (\frac{2}{3}\cdot\frac{4}{5} - \frac{2}{3}\cdot\frac{4}{5}\cdot\frac{6}{7}) + (\frac{2}{3}\cdot\frac{4}{5}\cdot\frac{6}{7} - \frac{2}{3}\cdot\frac{4}{5}\cdot\frac{6}{7}\cdot\frac{8}{9}) + \frac{2}{3}\cdot\frac{4}{5}\cdot\frac{6}{7}\cdot\frac{8}{9} + \cdots\\
\end{gather*}
After the second term, every subsequent pair of terms cancels each other out, leaving behind $\frac{1}{3} + \frac{2}{3} = 1$. Therefore, state 1 is recurrent. To further investigate if it is positively recurrent or null recurrent, we compute mean recurrence time $\mu$. Recall that every odd time-step probability of first return is 0, so we will only consider the even terms.
\begin{gather*}
\mu=\sum_{n=1}^{\infty}nf_{1}^{(n)} =  2\cdot\frac{1}{3} + 4\cdot(\frac{2}{3}\cdot\frac{1}{5}) + 6\cdot(\frac{2}{3}\cdot\frac{4}{5}\cdot\frac{1}{7}) + 8\cdot(\frac{2}{3}\cdot\frac{4}{5}\cdot\frac{6}{7}\cdot\frac{1}{9}) + \cdots\\
\mu=\sum_{n=1}^{\infty}nf_{1}^{(n)} =  \frac{2}{3} + (\frac{4}{3}\cdot\frac{1}{5}) + (\frac{2}{3}\cdot\frac{4}{5}\cdot\frac{6}{7}) + (\frac{2}{3}\cdot\frac{4}{5}\cdot\frac{6}{7}\cdot\frac{8}{9}) + \cdots
\end{gather*}
The $nth$ term can be represented as $t_n = \prod_{i=1}^{n}(1-a_i)$, where $a_n = (2n+1)^{-1}$

As a series, we can rewrite $t_n$
\begin{gather*}
t_n = \frac{2}{3}\cdot\frac{4}{5}\cdot\frac{6}{7}\cdots(1-a_n) = \frac{2}{3}\cdot\frac{4}{5}\cdot\frac{6}{7}\cdots(1-\frac{1}{2n+1}) = \frac{2}{3}\cdot\frac{4}{5}\cdot\frac{6}{7}\cdots\frac{2n}{2n+1}\\
\implies t_n =  2^n\frac{1}{3}\cdot\frac{2}{5}\cdot\frac{3}{7}\cdots\frac{n}{2n+1} = \frac{2^n n!}{3\cdot5\cdot7\cdots(2n+1)} =  \frac{2^n n!\cdot2\cdot4\cdot6\cdots2n}{1\cdot2\cdot3\cdot4\cdot5\cdot6\cdot7\cdots2n\cdot(2n+1)}\\
\implies t_n =  \frac{2^n n!\cdot2^n(1\cdot2\cdot3\cdot4\cdots n)}{1\cdot2\cdot3\cdot4\cdot5\cdot6\cdot7\cdots2n\cdot(2n+1)} = \frac{2^n n!\cdot 2^n n!}{(2n+1)!} = \frac{(2^n n!)^2}{(2n+1)!}
\end{gather*}
Expanding $t_n$, we can write
\begin{gather*}
t_n = \frac{(2^n n!)^2}{(2n+1)!} = 2^{2n}\frac{(n!)^2}{(2n+1)!} = 4^{n}\frac{(n!)^2}{(2n+1)!}\\
\implies t_n = 4^n \frac{n^2}{(2n+1)(2n)}\cdot\frac{(n-1)^2}{(2n-1)(2n-2)}\cdot\frac{(n-2)^2}{(2n-3)(2n-4)}\cdots\frac{2^2}{5\cdot4}\cdot\frac{1^2}{3\cdot2}\\
\implies t_n = 4^n \frac{n^2}{(2n+1)(2n)}\cdot\frac{(n-1)^2}{(2n-1)(2(n-1))}\cdot\frac{(n-2)^2}{(2n-3)(2(n-2))}\cdots\frac{2^2}{5\cdot2\cdot2}\cdot\frac{1^2}{3\cdot2\cdot1}\\
\implies t_n = 4^n \frac{n}{(2n+1)(2)}\cdot\frac{(n-1)}{(2n-1)(2)}\cdot\frac{(n-2)}{(2n-3)(2)}\cdots\frac{2}{5\cdot2}\cdot\frac{1}{3\cdot2}\\
\implies t_n = 4^n \frac{n}{(2(n+\frac{1}{2}))(2)}\cdot\frac{(n-1)}{(2(n-\frac{1}{2}))(2)}\cdot\frac{(n-2)}{(2(n-\frac{3}{2}))(2)}\cdots\frac{2}{2\cdot\frac{5}{2}\cdot2}\cdot\frac{1}{2\cdot\frac{3}{2}\cdot2}\\
\implies t_n = \frac{4^n}{4^n} \frac{n}{(n+\frac{1}{2})}\cdot\frac{(n-1)}{(n-\frac{1}{2})}\cdot\frac{(n-2)}{(n-\frac{3}{2})}\cdots\frac{2}{\frac{5}{2}}\cdot\frac{1}{\frac{3}{2}}\\
\implies t_n = \frac{n}{(n+\frac{1}{2})}\cdot\frac{(n-1)}{(n-\frac{1}{2})}\cdot\frac{(n-2)}{(n-\frac{3}{2})}\cdots\frac{2}{\frac{5}{2}}\cdot\frac{1}{\frac{3}{2}}
\end{gather*}
Subtracting $\frac{1}{2}$ from all the numerators inevitably leaves us with a term smaller than $t_n$. Therefore,
\begin{gather*}
t_n \ge\frac{n-\frac{1}{2}}{(n+\frac{1}{2})}\cdot\frac{(n-\frac{3}{2})}{(n-\frac{1}{2})}\cdot\frac{(n-\frac{5}{2})}{(n-\frac{3}{2})}\cdots\frac{\frac{3}{2}}{\frac{5}{2}}\cdot\frac{\frac{1}{2}}{\frac{3}{2}}\\
\implies t_n \ge \frac{\frac{1}{2}}{n+\frac{1}{2}}
\end{gather*}
The last line follows from alternate numerators and denominators cancelling out. We now have $t_n \ge \frac{1}{2n+1}$, which can be compared against the harmonic series to showcase divergence. By the limit comparison test, for two given series $\sum a_n$ and $\sum b_n$, if $\lim_{n\to\infty} \frac{a_n}{b_n} = c$ for $0<c<\infty$, where $a_n,b_n$ are respectively the $nth$ terms of the series, then either both series converge or both series diverge.

We have $a_n = \frac{1}{2n+1}, \ b_n = \frac{1}{n}$, the latter denoting the harmonic series.
\begin{gather*}
\lim_{n\to\infty} \frac{a_n}{b_n} = \lim_{n\to\infty} \frac{\frac{1}{2n+1}}{\frac{1}{n}} = \lim_{n\to\infty} \frac{n}{2n+1} = \frac{1}{2}, \text{L'Hospital rule}
\end{gather*}
As we know that the harmonic series diverges, the series with $nth$ term $a_n = \frac{1}{2n+1}$ also diverges. Therefore the series with nth term $t_n$, which is greater than $a_n$, must also diverge. Hence, the mean recurrence time $\mu$ diverges. This proves that the state 1 is null recurrent \textbf{Answer}.