\textbf{1}. Consider customers arriving independently at a bar that has two bartenders serving independently. Customers are assigned to an available bartender on a first come, first served basis
but walk away if there are more than $K > 1$ customers waiting or being served at the bar. You
may assume that arrivals are a Poisson process with rate $\lambda$ and each of the bartenders serves
customers according to a Poisson process with rate $\mu$.

(a) 
The Kendall notation for the above Markov chain is M/M/2
\begin{figure}[H]
\centering
\includegraphics[scale=0.5]{M-chains.jpg}
\caption{\label{fig:M-chains}Diagram for Markov Chain. Numbers represent size of the system, i.e. how many people there are in the cafe}
\end{figure}

(b) Let $\rho = \frac{\lambda}{2\mu}$. The steady state probability equations are as follows:
\begin{gather*}
\lambda P_0  = \mu P_1\ \implies P_1 = \frac{\lambda}{\mu}P_0 = 2\rho P_0\\
\lambda P_1 = 2\mu P_2\ \implies P_2 = \frac{\lambda}{2\mu}P_1 = 2\rho^2 P_0\\
...\\
\lambda P_{n-1} = 2\mu P_{n}\ \implies P_n = \frac{\lambda}{2\mu}P_{n-1} = 2\rho^n P_0\\
...\\
\lambda P_{K-1} = 2\mu P_{K}\ \implies P_K = \frac{\lambda}{2\mu}P_{K-1} = 2\rho^K P_0
\end{gather*}

We know that the sum of probabilities must add up to 1.
\begin{gather*}
P_0 + P_1 + ... + P_K = 1\\
\implies P_0(1 + 2\rho + 2\rho^2 + ... + 2\rho^K) = 1\\
\implies P_0(2 + 2\rho + 2\rho^2 + ... + 2\rho^K-1) = 1\\
\implies P_0(2(1 + \rho + \rho^2 + ... + \rho^K) - 1) = 1\\
\implies P_0(2\cdot \frac{1 - \rho^{K+1}}{1-\rho} - 1) = 1, \ \text{using given hint} \\
\implies P_0(2 - 2\rho^{K+1} - 1 + \rho) = 1 - \rho\\
\implies P_0(1 + \rho - 2\rho^{K+1}) = 1 - \rho\\
\implies P_0 = \frac{1-\rho}{1+\rho - 2\rho^{K+1}}\\
\implies P_n = 2\rho^n\frac{1-\rho}{1+\rho - 2\rho^{K+1}},\ \text{for $n\ge1$}
\end{gather*}

(c) At steady state, expected number of customers at the bar is given by E[N].
\begin{gather*}
E[N] = \sum_{n=0}^{K}nP_n\\
E[N] = \sum_{n=0}^{K}2n\rho^n\frac{1-\rho}{1+\rho - 2\rho^{K+1}}\\
E[N] = \frac{2(1-\rho)}{1+\rho - 2\rho^{K+1}}\sum_{n=0}^{K}n\rho^n\\
E[N] = \frac{2(1-\rho)}{1+\rho - 2\rho^{K+1}}[\rho + 2\rho^2 + 3\rho^3 + ... + K\rho^K]\\
E[N] = \frac{2(1-\rho)}{1+\rho - 2\rho^{K+1}}[\rho \cdot \frac{d}{d\rho} (1 + \rho + \rho^2 + \rho^3 + ... + \rho^K)]\\
E[N] = \frac{2\rho(1-\rho)}{1+\rho - 2\rho^{K+1}}\frac{d}{d\rho} (\frac{1 - \rho^{K+1}}{1 - \rho}), \ \text{using given hint as K is a positive integer} \\
E[N] = \frac{2\rho(1-\rho)}{1+\rho - 2\rho^{K+1}}\frac{(1-\rho)(-(K+1)\rho^K) - (1 - \rho^{K+1})(-1)}{(1-\rho)^2}\\
E[N] = \frac{2\rho(1-\rho)}{1+\rho - 2\rho^{K+1}}\frac{-(K+1)\rho^K + (K+1)\rho^{K+1}+1 - \rho^{K+1}}{(1-\rho)^2}\\
E[N] = \frac{2\rho}{1+\rho - 2\rho^{K+1}}\frac{1 - (K+1)\rho^K + K\rho^{K+1}}{(1-\rho)}
\end{gather*}

When $K\to\infty$ and $\rho < 1$,
\begin{gather*}
\lim_{K\to\infty}E[N] = \lim_{K\to\infty} \frac{2\rho}{1+\rho - 2\rho^{K+1}}\frac{1 - (K+1)\rho^K + K\rho^{K+1}}{(1-\rho)}\\
\lim_{K\to\infty}E[N] = \frac{2\rho}{1+\rho - 0}\frac{1 - 0 + 0}{(1-\rho)}
\end{gather*}
Note: We see terms with $\rho^{K}$ or $\rho^{K+1}$ becoming 0 as such terms drop to 0 exponentially fast thus offsetting their coefficients, which tend to inifinity. Thus, we see that for $K\to\infty$ and $\rho<1$, $E[N]\to\frac{2\rho}{1-\rho^2}$

(d) Using Little's Law, we find the average waiting time as $W_s = \frac{E[N]}{\lambda_{eff}}$. To derive an expression for $\lambda_{eff}$, we have
\begin{gather*}
\lambda_{eff} = \sum_{n=0}^{K}\lambda_n P_n = \sum_{n=0}^{K-1}\lambda_n P_n = \lambda(1-P_K),\ \text{(as $\lambda_K = 0)$}\\
\lambda_{eff} = \lambda \left[1 - \frac{2\rho^K(1-\rho)}{1 + \rho - 2\rho^{K+1}} \right]\\
\lambda_{eff} = \lambda \left[\frac{1 + \rho - 2\rho^{K+1} - 2\rho^K +2\rho^{K+1}}{1 + \rho - 2\rho^{K+1}} \right]\\
\lambda_{eff} = \lambda \left[\frac{1 + \rho - 2\rho^K}{1 + \rho - 2\rho^{K+1}} \right]
\end{gather*}
(e) Probability at steady state of having 2 or more customers at the bar is found as
\begin{gather*}
P(N\ge2) = P(N=2) + P(N=3) + ... + P(N=k)\\
=P_2 + P_3 + ... + P_K\\
=P_0(2\rho^2 + 2\rho^3 + ...+ 2\rho^K)\\
=\frac{(1-\rho)}{1 + \rho - 2\rho^{K+1}}(2\rho^2 + 2\rho^3 + ... + 2\rho^K)\\
=\frac{(1-\rho)}{1 + \rho - 2\rho^{K+1}}\cdot 2\rho^2(1 + \rho + ...+ \rho^{K-2})\\
=\frac{2(1-\rho)\rho^2}{1 + \rho - 2\rho^{K+1}}\cdot\left[\frac{1-\rho^{K-1}}{1-\rho}\right]\\
=\frac{2\rho^2}{1 + \rho - 2\rho^{K+1}}\cdot\left[1-\rho^{K-1}\right]
\end{gather*}

When $K\to\infty$,
\begin{gather*}
\lim_{K\to\infty}P(N\ge2) = \frac{2\rho^2}{1 + \rho} < \frac{1}{3}\\
\implies 6\rho^2 < 1 + \rho\\
\implies 6\rho^2 - \rho - 1< 0
\end{gather*}

Solving the above, we get $\frac{-1}{3}<\rho<\frac{1}{2}$. However, the first part of this is non-sensical as $\rho$ cannot be less than 0 (both $\lambda$ and $\mu$ are positive). Taking that into account, the possible range of values for $\rho$ is given as $0 < \rho < \frac{1}{2}$.