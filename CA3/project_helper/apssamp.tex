% ****** Start of file apssamp.tex ******
%
%   This file is part of the APS files in the REVTeX 4.2 distribution.
%   Version 4.2a of REVTeX, December 2014
%
%   Copyright (c) 2014 The American Physical Society.
%
%   See the REVTeX 4 README file for restrictions and more information.
%
% TeX'ing this file requires that you have AMS-LaTeX 2.0 installed
% as well as the rest of the prerequisites for REVTeX 4.2
%
% See the REVTeX 4 README file
% It also requires running BibTeX. The commands are as follows:
%
%  1)  latex apssamp.tex
%  2)  bibtex apssamp
%  3)  latex apssamp.tex
%  4)  latex apssamp.tex
%
\documentclass[%
 reprint,
 amsmath,amssymb,
 aps,
]{revtex4-2}

\usepackage{graphicx}% Include figure files
\usepackage{dcolumn}% Align table columns on decimal point
\usepackage{bm}% bold math
\usepackage{amsthm}%theorems and stuff
%\usepackage{algpseudocode}%algorithm writing
%\usepackage{algorithm}% algorithm writing
\usepackage[ruled,lined]{algorithm2e}
\usepackage{svg}
\usepackage{float}
%\usepackage{hyperref}% add hypertext capabilities
%\usepackage[mathlines]{lineno}% Enable numbering of text and display math
%\linenumbers\relax % Commence numbering lines
\setlength\parindent{0pt} % zero indent
\theoremstyle{definition}
\newtheorem{definition}{Definition}[section]


%\usepackage[showframe,%Uncomment any one of the following lines to test 
%%scale=0.7, marginratio={1:1, 2:3}, ignoreall,% default settings
%%text={7in,10in},centering,
%%margin=1.5in,
%%total={6.5in,8.75in}, top=1.2in, left=0.9in, includefoot,
%%height=10in,a5paper,hmargin={3cm,0.8in},
%]{geometry}

\begin{document}

\preprint{APS/123-QED}

\title{ECMM450 Stochastic Processes\\Simulation of Non-Homogeneous Poisson Processes}% Force line breaks with \\
%\thanks{A footnote to the article title}%

\author{70054986}
 \affiliation{%
 Department of Computer Science, University of Exeter
}%

%\collaboration{MUSO Collaboration}%\noaffiliation

%\author{Charlie Author}
 %\homepage{http://www.Second.institution.edu/~Charlie.Author}
%\affiliation{
%Second institution and/or address\\
% This line break forced% with \\
%}%
%\affiliation{
% Third institution, the second for Charlie Author
%}%
%\author{Delta Author}
%\affiliation{%
% Authors' institution and/or address\\
% This line break forced with \textbackslash\textbackslash
%}%

%\collaboration{CLEO Collaboration}%\noaffiliation

\date{\today}% It is always \today, today,
             %  but any date may be explicitly specified

\begin{abstract}
This project is about Non-Homogeneous Poisson processes and how to simulate them. We will be reviewing the thinning algorithm of Lewis and Shedler (1979) for simulating NHPP.
%\begin{description}
%\item[Usage]
%Secondary publications and information retrieval purposes.
%\item[Structure]
%You may use the \texttt{description} environment to structure your abstract;
%use the optional argument of the \verb+\item+ command to give the category of each item. 
%\end{description}
\end{abstract}

%\keywords{Suggested keywords}%Use showkeys class option if keyword
                              %display desired
\maketitle

%\tableofcontents

\section{\label{sec:level1}Introduction}

This report will be loosely structured according to the following points:
\begin{enumerate}
\item Explanataion of what is meant by a Non-Homogeneous Poisson Process (NHPP) accompanied by clear mathematical definition.
\item Review of the thinning algorithm of Lewis and Shedler (1979) for simulating NHPP. Short description of the algorithm explaining briefly why it works, and its main benefits compared to other approaches.
\item Simulation of occurrence of 1000 successive events ${t_1, t_2, . . . , t_{1000}}$ for a homogeneous Poisson process having a rate of 8 events per year. Visual representation of the same.
%Denoting the number of events in the time interval $[0, t]$ by $N(t)$, make a figure showing $N(t_i)$ versus $t_i$ for $i = 1, 2, . . . , 1000$ and comment on the main features that can be seen.
\item Considering a NHPP that has a rate function that increases smoothly from 1 event per year at $t = 0$ by 1\% per year, i.e. $(t) = (1.01)^t$. Determining whether the thinning algorithm can be used to simulate this process from the previous homogeneous Poisson process data.
\item Code to perform the thinning algorithm and use it to find occurence times for a NHPP having the rate function $\lambda(t) = (1.01)^t$.
\item Make a figure showing $N(t)$ versus $t$ for your NHPP simulation and compare it to what was shown in the figure for the homogeneous Poisson process. By integrating the rate function, add a line to your figure showing the expectation $E[N(t)]$ versus $t$.
\end{enumerate}

\section{\label{q:1}Description and Mathematical Definition}
The following section details what is meant by a Non-Homogeneous Poisson Process (NHPP) giving a clear
mathematical definition.\\

A non-homogeneous Poisson process can be thought of as a generalization of the homogeneous Poisson process, in that, as opposed to its homogeneous counterpart where rate of occurrence of events is constant (denoted by $\lambda$), here the rate is a function of time, denoted by $\lambda(t)$. Thus, the number of occrrences in the interval $(0,T]$ follows Poisson distribution $Pois(\int_{0}^{T}\lambda(t)dt)$. More formally, allowing the rate parameter to vary with time results in the following definition.
\begin{definition}[Ross, 2009, p.339, Definition 5.4] The counting process ${N(t), t \ge 0}$ is said to be a nonhomogeneous Poisson process with intensity function $\lambda(t) \ge 0, t \ge 0$, if
\begin{enumerate}
\item $N(0) = 0$.
\item The process has independent increments.
\item $P{N(t + h) - N(h) = 1} = \lambda(t)h + o(h)$.
\item $P{N(t + h) - N(h) \ge 2} = o(h)$.
\end{enumerate}
where $o(h)$ denotes higher order terms of $h$
\end{definition}

Time sampling an ordinary Poisson process (with constant rate $\lambda$) results in a Non-Homogeneous Poisson process. Given $\{N(t), t \ge 0\}$, a Poisson process wih rate $\lambda$, if the event occurring at time $t$ is counted with probability $p(t)$, then $\{N_c(t) , t\ge 0\}$ is a Non-Homogeneous Poisson process. <Insert reference>

\section{\label{q:2}Review of the Thinning Algorithm}


\subsubsection{Some other other algorithms}
There exist other algorithms for simulating Poisson processes, some of which we will discuss now.
\begin{enumerate}
\item Time-scale transformation of a homogeneous Poisson process via inverse of the integrated rate function $\Lambda(x)$
\item Generate intervals between the points individually
\item Order statistics from Poisson variates
\item Log-linear rate function
\end{enumerate}
\subsubsection{The thinning algorithm}
To construct a Non-Homogeneous Poisson process $\{N(t) , t \ge 0 \}$, with rate parameter $\lambda(t)$, over the interval $(0,T]$, the algorithm starts with a Non-Homogeneous Poisson process $\{N^{*}(t) , t \ge 0 \}$, with rate parameter $\bar{\lambda}(t)$ that dominates the set $\lambda(t)$ for all $t \in (0,T]$, that is 
\begin{gather*}
\bar{\lambda}(t) \ge \lambda(t) \forall t \in (0,T]\\
\bar{\lambda}(t) = sup_{t\in(0,T]}\lambda(t)
\end{gather*}

Then, for all $t$, the point from the dominating NHPP is retained with probability $\lambda(t)/\bar{\lambda}$. The remaining points form a NHPP with rate parameter $\lambda(t)$. It is noted that since points are deleted independently, the number of points in $\{N(x) : x \ge 0\}$ in any set of non-overlapping intervals are mutually independent.

\begin{algorithm}
\label{alg:1}
\caption{(Lewis and Shedler, 1979, p.7, Algorithm 1) Simulation of an Inhomogeneous Poisson Process with Bounded Intensity Function $\lambda(t)$, on $[0, T]$}
\textbf{Input}: $\lambda,T$

Initialize $n = m = 0$, $t_0 = s_0 = 0$, $\bar{\lambda}(t) = sup_{t\in(0,T]}\lambda(t)$;

\While{$s_m < T$}{
Generate $u\sim$\texttt{uniform(0,1)}\; 
Let $w = -ln (u) / \bar{\lambda}$\; %\rcomment{// so that $w\sim Expo(\bar{\lambda})$}
Set $s_{m+1} = s_m + w$\;
Generate $D\sim$ \texttt{uniform(0,1)}\;
\If{$D < \lambda(s_{m+1})/\bar{\lambda}$}{
   $t_{n+1} = s_{m+1}$\;
   $n = n+1$\;
   }
$m = m + 1$
}
\eIf{$t_n \le T$}{
\Return $\{t_k\}_{k=1,2,...,n}$
}{
\Return $\{t_k\}_{k=1,2,...,n-1}$
}
\end{algorithm}

\subsubsection{Benefits over other algorithms}
The paper discusses a few other methods with which to simulate NHPP. However, each method entails drawbacks with respect to computational efficiency. in its simplest implementation (<insert reference>), the thinning method obviates the need for numerical integration of the rate function, for ordering of points, and for generation of Poisson variates. 

\section{\label{q:3}Simulation of a Homogeneous Poisson process}

\begin{algorithm}
\label{alg:2}
\caption{Simulation of a Homogeneous Poisson Process with Rate $\lambda$, on $[0, T]$ <Insert reference to Yuanda Chen>}
\textbf{Input}: $\lambda,N$

Initialize $n_0 = 0 ,\  t_0 = 0$\;
\While{$True$}{
Generate $u\sim$\texttt{uniform(0,1)}\; 
Let $w = -ln (u) / \bar{\lambda}$\; %\rcomment{// so that $w\sim Expo(\bar{\lambda})$}
Set $t_{n+1} = t_n + w$\;
\eIf{$n+1 > N$}{
\Return $\{t_k\}_{k=1,2,...,n}$
}{
Set $n = n + 1 $\;
}
}
\end{algorithm}

Uniform random numbers are used to generate Poisson variates by using the following algorithm, that can be found in <Insert reference>. Full implementation in Python can be found in the Appendix. Figure \ref{fig:hpp} shows the result of simulating a homogeneous Poisson process.

\begin{figure}[H]
\centering
\includegraphics[scale=0.5]{HPP.png}
\caption{\label{fig:hpp}Simulation of the time of occurrence of 1000 successive events for a homogeneous Poisson process having a rate of 8 events per year. $N(t)$ denotes the number of events occurring in the time period $[0,t]$.}
\end{figure}

The observation is made that the graph appears relatively linear. This would be expected as the rate of occurrence of events remains constant over any interval of time $(a,b)$ and hence the slope corresponding to the number of events over time (i.e. rate) also remains the same.

\section{\label{q:4}NHPP with a smoothly increasing rate function}

A NHPP is considered with a rate function that increases smoothly from 1 event per year at $t = 0$ by $1\%$ per year, i.e. $\lambda(t) = (1.01)^t$. Looking at the previous homogeneous Poisson process from \ref{q:3}
, it can be seen that the maximum time taken to accumulate 1000 events is \~140. Figure \ref{fig:hpp_many_sim} shows this. Therefore, the maximum rate for the NHPP would be $(1.01)^{140} = 4.027$. Recall that the rate for the previous homogeneous Poisson process was 8. Hence, this is within the bounds of the original process. Thinning can be applied to simulate the NHPP from the previous HPP data.

\begin{figure}
\centering
\includegraphics[scale=0.5]{HPP_many_sim.png}
\caption{\label{fig:hpp_many_sim}Time in years to accumulate 1000 events at 8 per year}
\end{figure}

\section{\label{q:5}Perform the Thinning Algorithm with data from Homogeneous Poisson Process}
Using the thinning algorithm, the data points from the previous homogeneous process are time-sampled to construct a Non-Homogeneous process. 

\section{\label{q:6}Graphical Comparison of HPP and NHPP}
Figure \ref{fig:nhpp} showcases the cumulative events $N(t)$ against time $t$. It is observed that, as opposed to homogeneous Poisson process, the line is not quite linear - the slope increases as time progresses. This can be expected as the rate parameter $\lambda(t) = (1.01)^t$ also increases with time. Due to deletion of points from the original homogeneous process, it is also noted that the total number of points has decreased from 1000 to just over 300.

\begin{figure}
\centering
\includegraphics[scale=0.5]{NHPP.png}
\caption{\label{fig:nhpp}Accumulated events following a NHPP with rate parameter $\lambda(t) = 1.01^t$}
\end{figure}

By integrating the rate function, we get 
\begin{gather*}
\int_{0}^{t}\lambda(t)dt = \int_{0}^{t}1.01^tdt\\
=1.01^t/ln(1.01) + C
\end{gather*}

Noting the boundary condition that at $t=0$, $E[N(t)]=0$, we have $C = \frac{1.01^0}{ln(1.01)} = 100.499$. Thus,  $E[N(t)] = frac{1.01^0}{ln(1.01)} - 100.499$. Plotting this onto the earlier graph, Figure \ref{fig:nhppent} is obtained. It can be seen that the orange line closely follows the blue stepped graph.

\begin{figure}
\centering
\includegraphics[scale=0.5]{NHPP_ENt.png}
\caption{\label{fig:nhppent}$E[N(t)]$, denoted by $\frac{1.01^0}{ln(1.01)} - 100.499$ superimposed on the NHPP}
\end{figure}

Table~\ref{tab:table1},%
\begin{table}[b]%The best place to locate the table environment is directly after its first reference in text
\caption{\label{tab:table1}%
A table that fits into a single column of a two-column layout. 
Note that REV\TeX~4 adjusts the intercolumn spacing so that the table fills the
entire width of the column. Table captions are numbered
automatically. 
This table illustrates left-, center-, decimal- and right-aligned columns,
along with the use of the \texttt{ruledtabular} environment which sets the 
Scotch (double) rules above and below the alignment, per APS style.
}
\begin{ruledtabular}
\begin{tabular}{lcdr}
\textrm{Left\footnote{Note a.}}&
\textrm{Centered\footnote{Note b.}}&
\multicolumn{1}{c}{\textrm{Decimal}}&
\textrm{Right}\\
\colrule
1 & 2 & 3.001 & 4\\
10 & 20 & 30 & 40\\
100 & 200 & 300.0 & 400\\
\end{tabular}
\end{ruledtabular}
\end{table}
and Fig.~\ref{fig:epsart}.%
\begin{figure}[b]
\includegraphics{fig_1}% Here is how to import EPS art
\caption{\label{fig:epsart} A figure caption. The figure captions are
automatically numbered.}
\end{figure}

\section{Floats: Figures, Tables, Videos, etc.}
Figures and tables are usually allowed to ``float'', which means that their
placement is determined by \LaTeX, while the document is being typeset. 

Use the \texttt{figure} environment for a figure, the \texttt{table} environment for a table.
In each case, use the \verb+\caption+ command within to give the text of the
figure or table caption along with the \verb+\label+ command to provide
a key for referring to this figure or table.
The typical content of a figure is an image of some kind; 
that of a table is an alignment.%
\begin{figure*}
\includegraphics{fig_2}% Here is how to import EPS art
\caption{\label{fig:wide}Use the figure* environment to get a wide
figure that spans the page in \texttt{twocolumn} formatting.}
\end{figure*}
\begin{table*}
\caption{\label{tab:table3}This is a wide table that spans the full page
width in a two-column layout. It is formatted using the
\texttt{table*} environment. It also demonstates the use of
\textbackslash\texttt{multicolumn} in rows with entries that span
more than one column.}
\begin{ruledtabular}
\begin{tabular}{ccccc}
 &\multicolumn{2}{c}{$D_{4h}^1$}&\multicolumn{2}{c}{$D_{4h}^5$}\\
 Ion&1st alternative&2nd alternative&lst alternative
&2nd alternative\\ \hline
 K&$(2e)+(2f)$&$(4i)$ &$(2c)+(2d)$&$(4f)$ \\
 Mn&$(2g)$\footnote{The $z$ parameter of these positions is $z\sim\frac{1}{4}$.}
 &$(a)+(b)+(c)+(d)$&$(4e)$&$(2a)+(2b)$\\
 Cl&$(a)+(b)+(c)+(d)$&$(2g)$\footnotemark[1]
 &$(4e)^{\text{a}}$\\
 He&$(8r)^{\text{a}}$&$(4j)^{\text{a}}$&$(4g)^{\text{a}}$\\
 Ag& &$(4k)^{\text{a}}$& &$(4h)^{\text{a}}$\\
\end{tabular}
\end{ruledtabular}
\end{table*}

Insert an image using either the \texttt{graphics} or
\texttt{graphix} packages, which define the \verb+\includegraphics{#1}+ command.
(The two packages differ in respect of the optional arguments 
used to specify the orientation, scaling, and translation of the image.) 
To create an alignment, use the \texttt{tabular} environment. 

The best place to locate the \texttt{figure} or \texttt{table} environment
is immediately following its first reference in text; this sample document
illustrates this practice for Fig.~\ref{fig:epsart}, which
shows a figure that is small enough to fit in a single column. 

In exceptional cases, you will need to move the float earlier in the document, as was done
with Table~\ref{tab:table3}: \LaTeX's float placement algorithms need to know
about a full-page-width float earlier. 

Fig.~\ref{fig:wide}
has content that is too wide for a single column,
so the \texttt{figure*} environment has been used.%
\begin{table}[b]
\caption{\label{tab:table4}%
Numbers in columns Three--Five are aligned with the ``d'' column specifier 
(requires the \texttt{dcolumn} package). 
Non-numeric entries (those entries without a ``.'') in a ``d'' column are aligned on the decimal point. 
Use the ``D'' specifier for more complex layouts. }
\begin{ruledtabular}
\begin{tabular}{ccddd}
One&Two&
\multicolumn{1}{c}{\textrm{Three}}&
\multicolumn{1}{c}{\textrm{Four}}&
\multicolumn{1}{c}{\textrm{Five}}\\
%\mbox{Three}&\mbox{Four}&\mbox{Five}\\
\hline
one&two&\mbox{three}&\mbox{four}&\mbox{five}\\
He&2& 2.77234 & 45672. & 0.69 \\
C\footnote{Some tables require footnotes.}
  &C\footnote{Some tables need more than one footnote.}
  & 12537.64 & 37.66345 & 86.37 \\
\end{tabular}
\end{ruledtabular}
\end{table}

The content of a table is typically a \texttt{tabular} environment, 
giving rows of type in aligned columns. 
Column entries separated by \verb+&+'s, and 
each row ends with \textbackslash\textbackslash. 
The required argument for the \texttt{tabular} environment
specifies how data are aligned in the columns. 
For instance, entries may be centered, left-justified, right-justified, aligned on a decimal
point. 
Extra column-spacing may be be specified as well, 
although REV\TeX~4 sets this spacing so that the columns fill the width of the
table. Horizontal rules are typeset using the \verb+\hline+
command. The doubled (or Scotch) rules that appear at the top and
bottom of a table can be achieved enclosing the \texttt{tabular}
environment within a \texttt{ruledtabular} environment. Rows whose
columns span multiple columns can be typeset using the
\verb+\multicolumn{#1}{#2}{#3}+ command (for example, see the first
row of Table~\ref{tab:table3}).%

Tables~\ref{tab:table1}, \ref{tab:table3}, \ref{tab:table4}, and \ref{tab:table2}%
\begin{table}[b]
\caption{\label{tab:table2}
A table with numerous columns that still fits into a single column. 
Here, several entries share the same footnote. 
Inspect the \LaTeX\ input for this table to see exactly how it is done.}
\begin{ruledtabular}
\begin{tabular}{cccccccc}
 &$r_c$ (\AA)&$r_0$ (\AA)&$\kappa r_0$&
 &$r_c$ (\AA) &$r_0$ (\AA)&$\kappa r_0$\\
\hline
Cu& 0.800 & 14.10 & 2.550 &Sn\footnotemark[1]
& 0.680 & 1.870 & 3.700 \\
Ag& 0.990 & 15.90 & 2.710 &Pb\footnotemark[2]
& 0.450 & 1.930 & 3.760 \\
Au& 1.150 & 15.90 & 2.710 &Ca\footnotemark[3]
& 0.750 & 2.170 & 3.560 \\
Mg& 0.490 & 17.60 & 3.200 &Sr\footnotemark[4]
& 0.900 & 2.370 & 3.720 \\
Zn& 0.300 & 15.20 & 2.970 &Li\footnotemark[2]
& 0.380 & 1.730 & 2.830 \\
Cd& 0.530 & 17.10 & 3.160 &Na\footnotemark[5]
& 0.760 & 2.110 & 3.120 \\
Hg& 0.550 & 17.80 & 3.220 &K\footnotemark[5]
&  1.120 & 2.620 & 3.480 \\
Al& 0.230 & 15.80 & 3.240 &Rb\footnotemark[3]
& 1.330 & 2.800 & 3.590 \\
Ga& 0.310 & 16.70 & 3.330 &Cs\footnotemark[4]
& 1.420 & 3.030 & 3.740 \\
In& 0.460 & 18.40 & 3.500 &Ba\footnotemark[5]
& 0.960 & 2.460 & 3.780 \\
Tl& 0.480 & 18.90 & 3.550 & & & & \\
\end{tabular}
\end{ruledtabular}
\footnotetext[1]{Here's the first, from Ref.~\onlinecite{feyn54}.}
\footnotetext[2]{Here's the second.}
\footnotetext[3]{Here's the third.}
\footnotetext[4]{Here's the fourth.}
\footnotetext[5]{And etc.}
\end{table}
show various effects.
A table that fits in a single column employs the \texttt{table}
environment. 
Table~\ref{tab:table3} is a wide table, set with the \texttt{table*} environment. 
Long tables may need to break across pages. 
The most straightforward way to accomplish this is to specify
the \verb+[H]+ float placement on the \texttt{table} or
\texttt{table*} environment. 
However, the \LaTeXe\ package \texttt{longtable} allows headers and footers to be specified for each page of the table. 
A simple example of the use of \texttt{longtable} can be found
in the file \texttt{summary.tex} that is included with the REV\TeX~4
distribution.

There are two methods for setting footnotes within a table (these
footnotes will be displayed directly below the table rather than at
the bottom of the page or in the bibliography). The easiest
and preferred method is just to use the \verb+\footnote{#1}+
command. This will automatically enumerate the footnotes with
lowercase roman letters. However, it is sometimes necessary to have
multiple entries in the table share the same footnote. In this case,
there is no choice but to manually create the footnotes using
\verb+\footnotemark[#1]+ and \verb+\footnotetext[#1]{#2}+.
\texttt{\#1} is a numeric value. Each time the same value for
\texttt{\#1} is used, the same mark is produced in the table. The
\verb+\footnotetext[#1]{#2}+ commands are placed after the \texttt{tabular}
environment. Examine the \LaTeX\ source and output for
Tables~\ref{tab:table1} and \ref{tab:table2}
for examples.

Video~\ref{vid:PRSTPER.4.010101} 
illustrates several features new with REV\TeX4.2,
starting with the \texttt{video} environment, which is in the same category with
\texttt{figure} and \texttt{table}.%
\begin{video}
\href{http://prst-per.aps.org/multimedia/PRSTPER/v4/i1/e010101/e010101_vid1a.mpg}{\includegraphics{vid_1a}}%
 \quad
\href{http://prst-per.aps.org/multimedia/PRSTPER/v4/i1/e010101/e010101_vid1b.mpg}{\includegraphics{vid_1b}}
 \setfloatlink{http://link.aps.org/multimedia/PRSTPER/v4/i1/e010101}%
 \caption{\label{vid:PRSTPER.4.010101}%
  Students explain their initial idea about Newton's third law to a teaching assistant. 
  Clip (a): same force.
  Clip (b): move backwards.
 }%
\end{video}
The \verb+\setfloatlink+ command causes the title of the video to be a hyperlink to the
indicated URL; it may be used with any environment that takes the \verb+\caption+
command.
The \verb+\href+ command has the same significance as it does in the context of
the \texttt{hyperref} package: the second argument is a piece of text to be 
typeset in your document; the first is its hyperlink, a URL.

\textit{Physical Review} style requires that the initial citation of
figures or tables be in numerical order in text, so don't cite
Fig.~\ref{fig:wide} until Fig.~\ref{fig:epsart} has been cited.

\begin{acknowledgments}
We wish to acknowledge the support of the author community in using
REV\TeX{}, offering suggestions and encouragement, testing new versions,
\dots.
\end{acknowledgments}

\appendix


\section{Code}
Below is the code for section \ref{q:3}
\begin{verbatim}
# Imports
import numpy as np
import matplotlib.pyplot as plt

# define hpp function
def gen_hpp(lmbda ,  N):
    '''
    param lmbda: rate parameter
    N: Number of events
    '''
    # inits
    t = [0]

    # begin loop
    while True:
        # generate uniform r.v. ~ Unif[0,1]
        u = np.random.uniform(0,1) 
        # generate w ~ Exponential(lambda)
        w = - np.log(u)/lmbda 
        t.append(t[-1] + w)
        # exit condition
        if len(t) > N:
        	  # get time to event & count the 
        	  # number of events
           return t , np.arange(len(t)) 

if __name__ == '__main__': # main namespace
    l , N = 8 , 1000
    # generate the time(s) to event(s) 
    # AND count of events 
    hpp_event_times , events = \
    gen_hpp(lmbda = l , N = N) 
    print(events , hpp_event_times) # debug
    
    # Make plots
    fig , ax = plt.subplots()
    # step graph
    ax.step(hpp_event_times , events , 
    		label = f"$\lambda$ = {l}" , lw = 0.5) 
    ax.set_xlabel(r'$t$')
    ax.set_ylabel(r'$N(t)$')
    ax.set_title('Homogeneous Poisson process')
    ax.legend(loc='best')
    plt.show()
\end{verbatim}

Below is the code for NHPP in section \ref{q:5}
\begin{verbatim}

# Imports
import numpy as np
import matplotlib.pyplot as plt

# define hpp function
def gen_nhpp(lmbda_bar = 8 ,  T = 140):
    '''
    param lmbda_bar: rate parameter that 
    dominates the rate param of the NHPP
    T : maximum time
    '''
    # inits
    s = [0]
    t = [0]

    # begin loop
    while s[-1] < T:
        # generate uniform r.v. ~ Unif[0,1]
        u = np.random.uniform(0,1)
        # generate w ~ Exponential(lambda) 
        w = - np.log(u)/lmbda_bar 
        s.append(s[-1] + w)
        # geenrate D ~ Unif[0,1]
        D = np.random.uniform(0,1)
        # acceptance criterion
        if D < (1.01)**s[-1] / lmbda_bar:
            t.append(s[-1])
        
        if t[-1] > T:
            num_events = np.arange(len(t[:-1]))
            print(f'Breakpoint 1: the number of events \
            is {num_events[-1]}, and the time taken to \
            reach them is {t[:-1][-1]}')
            # get time to event & count the 
            # number of events
            return t[:-1] , num_events
    else:
        num_events = np.arange(len(t))
        print(f'Breakpoint 2: the number of events is \
        {num_events[-1]}, and the time taken to reach \
        them is {t[-1]}')
        # get time to event & count the number 
        # of events
        return t , num_events

# main namespace
if __name__ == '__main__': 
    # generate the time(s) to event(s) 
    # AND count of events
    nhpp_event_times , events = gen_nhpp() 
    # Make plots
    fig , ax = plt.subplots()
    # step graph
    ax.step(nhpp_event_times , events , 
            label = f'$\lambda(t) = (1.01)^t$') 
    # Integrating the rate function to get E[N(t)]
    x = np.arange(140)
    y = 1.01**x/np.log(1.01) - \
    (1.01)**0/np.log(1.01)
    ax.plot(x , y , 
    	label = r'E[N(t)] = $\frac{1.01^t}\
            {ln(1.01)}$ - 100.499')
    # Auxiliaries
    ax.set_xlabel(r'$t$')
    ax.set_ylabel(r'$N(t)$')
    ax.set_title('Non-Homogeneous \
                 Poisson process')
    ax.legend(loc='best')
    plt.show()

\end{verbatim}


\section{Appendixes}

To start the appendixes, use the \verb+\appendix+ command.
This signals that all following section commands refer to appendixes
instead of regular sections. Therefore, the \verb+\appendix+ command
should be used only once---to setup the section commands to act as
appendixes. Thereafter normal section commands are used. The heading
for a section can be left empty. For example,
\begin{verbatim}
\appendix
\section{}
\end{verbatim}
will produce an appendix heading that says ``APPENDIX A'' and
\begin{verbatim}
\appendix
\section{Background}
\end{verbatim}
will produce an appendix heading that says ``APPENDIX A: BACKGROUND''
(note that the colon is set automatically).

If there is only one appendix, then the letter ``A'' should not
appear. This is suppressed by using the star version of the appendix
command (\verb+\appendix*+ in the place of \verb+\appendix+).

\section{A little more on appendixes}

Observe that this appendix was started by using
\begin{verbatim}
\section{A little more on appendixes}
\end{verbatim}

Note the equation number in an appendix:
\begin{equation}
E=mc^2.
\end{equation}

\subsection{\label{app:subsec}A subsection in an appendix}

You can use a subsection or subsubsection in an appendix. Note the
numbering: we are now in Appendix~\ref{app:subsec}.

Note the equation numbers in this appendix, produced with the
subequations environment:
\begin{subequations}
\begin{eqnarray}
E&=&mc, \label{appa}
\\
E&=&mc^2, \label{appb}
\\
E&\agt& mc^3. \label{appc}
\end{eqnarray}
\end{subequations}
They turn out to be Eqs.~(\ref{appa}), (\ref{appb}), and (\ref{appc}).

% The \nocite command causes all entries in a bibliography to be printed out
% whether or not they are actually referenced in the text. This is appropriate
% for the sample file to show the different styles of references, but authors
% most likely will not want to use it.
\nocite{*}

\bibliography{apssamp}% Produces the bibliography via BibTeX.

\end{document}
%
% ****** End of file apssamp.tex ******
