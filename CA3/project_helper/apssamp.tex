% ****** Start of file apssamp.tex ******
%
%   This file is part of the APS files in the REVTeX 4.2 distribution.
%   Version 4.2a of REVTeX, December 2014
%
%   Copyright (c) 2014 The American Physical Society.
%
%   See the REVTeX 4 README file for restrictions and more information.
%
% TeX'ing this file requires that you have AMS-LaTeX 2.0 installed
% as well as the rest of the prerequisites for REVTeX 4.2
%
% See the REVTeX 4 README file
% It also requires running BibTeX. The commands are as follows:
%
%  1)  latex apssamp.tex
%  2)  bibtex apssamp
%  3)  latex apssamp.tex
%  4)  latex apssamp.tex
%
\documentclass[%
 reprint,
 amsmath,amssymb,
 aps,
]{revtex4-2}

\usepackage{graphicx}% Include figure files
\usepackage{dcolumn}% Align table columns on decimal point
\usepackage{bm}% bold math
\usepackage{amsthm}%theorems and stuff
%\usepackage{algpseudocode}%algorithm writing
%\usepackage{algorithm}% algorithm writing
\usepackage[ruled,lined]{algorithm2e}
\usepackage{svg}
\usepackage{float}
%\usepackage{hyperref}% add hypertext capabilities
%\usepackage[mathlines]{lineno}% Enable numbering of text and display math
%\linenumbers\relax % Commence numbering lines
\setlength\parindent{0pt} % zero indent
\theoremstyle{definition}
\newtheorem{definition}{Definition}[section]


%\usepackage[showframe,%Uncomment any one of the following lines to test 
%%scale=0.7, marginratio={1:1, 2:3}, ignoreall,% default settings
%%text={7in,10in},centering,
%%margin=1.5in,
%%total={6.5in,8.75in}, top=1.2in, left=0.9in, includefoot,
%%height=10in,a5paper,hmargin={3cm,0.8in},
%]{geometry}

\begin{document}

\preprint{APS/123-QED}

\title{ECMM450 Stochastic Processes\\Simulation of Non-Homogeneous Poisson Processes}% Force line breaks with \\
%\thanks{A footnote to the article title}%

\author{70054986}
 \affiliation{%
 Department of Computer Science, University of Exeter
}%

%\collaboration{MUSO Collaboration}%\noaffiliation

%\author{Charlie Author}
 %\homepage{http://www.Second.institution.edu/~Charlie.Author}
%\affiliation{
%Second institution and/or address\\
% This line break forced% with \\
%}%
%\affiliation{
% Third institution, the second for Charlie Author
%}%
%\author{Delta Author}
%\affiliation{%
% Authors' institution and/or address\\
% This line break forced with \textbackslash\textbackslash
%}%

%\collaboration{CLEO Collaboration}%\noaffiliation

\date{\today}% It is always \today, today,
             %  but any date may be explicitly specified

\begin{abstract}
This project is about Non-Homogeneous Poisson processes and how to simulate them. We will be reviewing the thinning algorithm of Lewis and Shedler (1979) for simulating NHPP.
%\begin{description}
%\item[Usage]
%Secondary publications and information retrieval purposes.
%\item[Structure]
%You may use the \texttt{description} environment to structure your abstract;
%use the optional argument of the \verb+\item+ command to give the category of each item. 
%\end{description}
\end{abstract}

%\keywords{Suggested keywords}%Use showkeys class option if keyword
                              %display desired
\maketitle

%\tableofcontents

\section{\label{sec:level1}Introduction}

This report will be loosely structured according to the following points:
\begin{enumerate}
\item Explanataion of what is meant by a Non-Homogeneous Poisson Process (NHPP) accompanied by clear mathematical definition.
\item Review of the thinning algorithm of Lewis and Shedler (1979) for simulating NHPP. Short description of the algorithm explaining briefly why it works, and its main benefits compared to other approaches.
\item Simulation of occurrence of 1000 successive events ${t_1, t_2, . . . , t_{1000}}$ for a homogeneous Poisson process having a rate of 8 events per year. Visual representation of the same.
%Denoting the number of events in the time interval $[0, t]$ by $N(t)$, make a figure showing $N(t_i)$ versus $t_i$ for $i = 1, 2, . . . , 1000$ and comment on the main features that can be seen.
\item Considering a NHPP that has a rate function that increases smoothly from 1 event per year at $t = 0$ by 1\% per year, i.e. $(t) = (1.01)^t$. Determining whether the thinning algorithm can be used to simulate this process from the previous homogeneous Poisson process data.
\item Code to perform the thinning algorithm and use it to find occurence times for a NHPP having the rate function $\lambda(t) = (1.01)^t$.
\item Make a figure showing $N(t)$ versus $t$ for your NHPP simulation and compare it to what was shown in the figure for the homogeneous Poisson process. By integrating the rate function, add a line to your figure showing the expectation $E[N(t)]$ versus $t$.
\end{enumerate}

\section{\label{q:1}Description and Mathematical Definition}
The following section details what is meant by a Non-Homogeneous Poisson Process (NHPP) giving a clear
mathematical definition.\\

A non-homogeneous Poisson process can be thought of as a generalization of the homogeneous Poisson process, in that, as opposed to its homogeneous counterpart where rate of occurrence of events is constant (denoted by $\lambda$), here the rate is a function of time, denoted by $\lambda(t)$. Thus, the number of occrrences in the interval $(0,T]$ follows Poisson distribution $Pois(\int_{0}^{T}\lambda(t)dt)$. More formally, allowing the rate parameter to vary with time results in the following definition.
\begin{definition}[Ross, 2009, p.339, Definition 5.4] The counting process ${N(t), t \ge 0}$ is said to be a nonhomogeneous Poisson process with intensity function $\lambda(t) \ge 0, t \ge 0$, if
\begin{enumerate}
\item $N(0) = 0$.
\item The process has independent increments.
\item $P{N(t + h) - N(h) = 1} = \lambda(t)h + o(h)$.
\item $P{N(t + h) - N(h) \ge 2} = o(h)$.
\end{enumerate}
where $o(h)$ denotes higher order terms of $h$
\end{definition}

Time sampling an ordinary Poisson process (with constant rate $\lambda$) results in a Non-Homogeneous Poisson process. Given $\{N(t), t \ge 0\}$, a Poisson process wih rate $\lambda$, if the event occurring at time $t$ is counted with probability $p(t)$, then $\{N_c(t) , t\ge 0\}$ is a Non-Homogeneous Poisson process. <Insert reference>

\section{\label{q:2}Review of the Thinning Algorithm}


\subsubsection{Some other other algorithms}
There exist other algorithms for simulating Poisson processes, some of which we will discuss now.
\begin{enumerate}
\item Time-scale transformation of a homogeneous Poisson process via inverse of the integrated rate function $\Lambda(x)$
\item Generate intervals between the points individually
\item Order statistics from Poisson variates
\item Log-linear rate function
\end{enumerate}
\subsubsection{The thinning algorithm}
To construct a Non-Homogeneous Poisson process $\{N(t) , t \ge 0 \}$, with rate parameter $\lambda(t)$, over the interval $(0,T]$, the algorithm starts with a Non-Homogeneous Poisson process $\{N^{*}(t) , t \ge 0 \}$, with rate parameter $\bar{\lambda}(t)$ that dominates the set $\lambda(t)$ for all $t \in (0,T]$, that is 
\begin{gather*}
\bar{\lambda}(t) \ge \lambda(t) \forall t \in (0,T]\\
\bar{\lambda}(t) = sup_{t\in(0,T]}\lambda(t)
\end{gather*}

Then, for all $t$, the point from the dominating NHPP is retained with probability $\lambda(t)/\bar{\lambda}$. The remaining points form a NHPP with rate parameter $\lambda(t)$. It is noted that since points are deleted independently, the number of points in $\{N(x) : x \ge 0\}$ in any set of non-overlapping intervals are mutually independent.

\begin{algorithm}
\label{alg:1}
\caption{(Lewis and Shedler, 1979, p.7, Algorithm 1) Simulation of an Inhomogeneous Poisson Process with Bounded Intensity Function $\lambda(t)$, on $[0, T]$}
\textbf{Input}: $\lambda,T$

Initialize $n = m = 0$, $t_0 = s_0 = 0$, $\bar{\lambda}(t) = sup_{t\in(0,T]}\lambda(t)$;

\While{$s_m < T$}{
Generate $u\sim$\texttt{uniform(0,1)}\; 
Let $w = -ln (u) / \bar{\lambda}$\; %\rcomment{// so that $w\sim Expo(\bar{\lambda})$}
Set $s_{m+1} = s_m + w$\;
Generate $D\sim$ \texttt{uniform(0,1)}\;
\If{$D < \lambda(s_{m+1})/\bar{\lambda}$}{
   $t_{n+1} = s_{m+1}$\;
   $n = n+1$\;
   }
$m = m + 1$
}
\eIf{$t_n \le T$}{
\Return $\{t_k\}_{k=1,2,...,n}$
}{
\Return $\{t_k\}_{k=1,2,...,n-1}$
}
\end{algorithm}

\subsubsection{Benefits over other algorithms}
The paper discusses a few other methods with which to simulate NHPP. However, each method entails drawbacks with respect to computational efficiency. in its simplest implementation (<insert reference>), the thinning method obviates the need for numerical integration of the rate function, for ordering of points, and for generation of Poisson variates. 

\section{\label{q:3}Simulation of a Homogeneous Poisson process}

\begin{algorithm}
\label{alg:2}
\caption{Simulation of a Homogeneous Poisson Process with Rate $\lambda$, on $[0, T]$ <Insert reference to Yuanda Chen>}
\textbf{Input}: $\lambda,N$

Initialize $n_0 = 0 ,\  t_0 = 0$\;
\While{$True$}{
Generate $u\sim$\texttt{uniform(0,1)}\; 
Let $w = -ln (u) / \bar{\lambda}$\; %\rcomment{// so that $w\sim Expo(\bar{\lambda})$}
Set $t_{n+1} = t_n + w$\;
\eIf{$n+1 > N$}{
\Return $\{t_k\}_{k=1,2,...,n}$
}{
Set $n = n + 1 $\;
}
}
\end{algorithm}

Uniform random numbers are used to generate Poisson variates by using the following algorithm, that can be found in <Insert reference>. Full implementation in Python can be found in the Appendix. Figure \ref{fig:hpp} shows the result of simulating a homogeneous Poisson process.

\begin{figure}[H]
\centering
\includegraphics[scale=0.5]{HPP.png}
\caption{\label{fig:hpp}Simulation of the time of occurrence of 1000 successive events for a homogeneous Poisson process having a rate of 8 events per year. $N(t)$ denotes the number of events occurring in the time period $[0,t]$.}
\end{figure}

The observation is made that the graph appears relatively linear. This would be expected as the rate of occurrence of events remains constant over any interval of time $(a,b)$ and hence the slope corresponding to the number of events over time (i.e. rate) also remains the same.

\section{\label{q:4}NHPP with a smoothly increasing rate function}

A NHPP is considered with a rate function that increases smoothly from 1 event per year at $t = 0$ by $1\%$ per year, i.e. $\lambda(t) = (1.01)^t$. Looking at the previous homogeneous Poisson process from \ref{q:3}
, it can be seen that the maximum time taken to accumulate 1000 events is \~140. Figure \ref{fig:hpp_many_sim} shows this. Therefore, the maximum rate for the NHPP would be $(1.01)^{140} = 4.027$. Recall that the rate for the previous homogeneous Poisson process was 8. Hence, this is within the bounds of the original process. Thinning can be applied to simulate the NHPP from the previous HPP data.

\begin{figure}
\centering
\includegraphics[scale=0.5]{HPP_many_sim.png}
\caption{\label{fig:hpp_many_sim}Time in years to accumulate 1000 events at 8 per year}
\end{figure}

\section{\label{q:5}Perform the Thinning Algorithm with data from Homogeneous Poisson Process}
Using the thinning algorithm, the data points from the previous homogeneous process are time-sampled to construct a Non-Homogeneous process. 

\section{\label{q:6}Graphical Comparison of HPP and NHPP}
Figure \ref{fig:nhpp} showcases the cumulative events $N(t)$ against time $t$. It is observed that, as opposed to homogeneous Poisson process, the line is not quite linear - the slope increases as time progresses. This can be expected as the rate parameter $\lambda(t) = (1.01)^t$ also increases with time. Due to deletion of points from the original homogeneous process, it is also noted that the total number of points has decreased from 1000 to just over 300.

\begin{figure}
\centering
\includegraphics[scale=0.5]{NHPP.png}
\caption{\label{fig:nhpp}Accumulated events following a NHPP with rate parameter $\lambda(t) = 1.01^t$}
\end{figure}

By integrating the rate function, we get 
\begin{gather*}
\int_{0}^{t}\lambda(t)dt = \int_{0}^{t}1.01^tdt\\
=1.01^t/ln(1.01) + C
\end{gather*}

Noting the boundary condition that at $t=0$, $E[N(t)]=0$, we have $C = \frac{1.01^0}{ln(1.01)} = 100.499$. Thus,  $E[N(t)] = \frac{1.01^0}{ln(1.01)} - 100.499$. Plotting this onto the earlier graph, Figure \ref{fig:nhppent} is obtained. It can be seen that the orange line closely follows the blue stepped graph.

\begin{figure}
\centering
\includegraphics[scale=0.5]{NHPP_ENt.png}
\caption{\label{fig:nhppent}$E[N(t)]$, denoted by $\frac{1.01^0}{ln(1.01)} - 100.499$ superimposed on the NHPP}
\end{figure}

\begin{acknowledgments}
We wish to acknowledge the support of the author community in using
REV\TeX{}, offering suggestions and encouragement, testing new versions,
\dots.
\end{acknowledgments}

\appendix

\section{Code}
Below is the code for section \ref{q:3}. It simulates a homogeneous Poisson process with rate parameter $\lambda=8$ per year. Time to 1000 events is simulated and the graph generated from this code is Figure \ref{fig:hpp}.
\begin{verbatim}
# Imports
import numpy as np
import matplotlib.pyplot as plt

# define hpp function
def gen_hpp(lmbda ,  N):
    '''
    param lmbda: rate parameter
    N: Number of events
    '''
    # inits
    t = [0]

    # begin loop
    while True:
        # generate uniform r.v. ~ Unif[0,1]
        u = np.random.uniform(0,1) 
        # generate w ~ Exponential(lambda)
        w = - np.log(u)/lmbda 
        t.append(t[-1] + w)
        # exit condition
        if len(t) > N:
        	  # get time to event & count the 
        	  # number of events
           return t , np.arange(len(t)) 

if __name__ == '__main__': # main namespace
    l , N = 8 , 1000
    # generate the time(s) to event(s) 
    # AND count of events 
    hpp_event_times , events = \
    gen_hpp(lmbda = l , N = N) 
    print(events , hpp_event_times) # debug
    
    # Make plots
    fig , ax = plt.subplots()
    # step graph
    ax.step(hpp_event_times , events , 
    		label = f"$\lambda$ = {l}" , lw = 0.5) 
    ax.set_xlabel(r'$t$')
    ax.set_ylabel(r'$N(t)$')
    ax.set_title('Homogeneous Poisson process')
    ax.legend(loc='best')
    plt.show()
\end{verbatim}

Below is the code for section \ref{q:5}. It simulates a Non-Homogeneous Poisson process with rate parameter $\lambda(t)=1.01^t$ that smoothly increases from 1 event per year. The time boundary is kept at 140 years as this is roughly the longest time it took the HPP to reach 1000 events. The graphs generated from this code are Figure \ref{fig:nhpp} and Figure \ref{fig:nhppent}. In the latter, $E[N(t)]$ is shown along with the realizations from the NHPP to show that both the lines adhere to each other.
\begin{verbatim}
# Imports
import numpy as np
import matplotlib.pyplot as plt

# define hpp function
def gen_nhpp(lmbda_bar = 8 ,  T = 140):
    '''
    param lmbda_bar: rate parameter that 
    dominates the rate param of the NHPP
    T : maximum time
    '''
    # inits
    s = [0]
    t = [0]

    # begin loop
    while s[-1] < T:
        # generate uniform r.v. ~ Unif[0,1]
        u = np.random.uniform(0,1)
        # generate w ~ Exponential(lambda) 
        w = - np.log(u)/lmbda_bar 
        s.append(s[-1] + w)
        # geenrate D ~ Unif[0,1]
        D = np.random.uniform(0,1)
        # acceptance criterion
        if D < (1.01)**s[-1] / lmbda_bar:
            t.append(s[-1])
        
        if t[-1] > T:
            num_events = np.arange(len(t[:-1]))
            print(f'Breakpoint 1: the number of events \
            is {num_events[-1]}, and the time taken to \
            reach them is {t[:-1][-1]}')
            # get time to event & count the 
            # number of events
            return t[:-1] , num_events
    else:
        num_events = np.arange(len(t))
        print(f'Breakpoint 2: the number of events is \
        {num_events[-1]}, and the time taken to reach \
        them is {t[-1]}')
        # get time to event & count the number 
        # of events
        return t , num_events

# main namespace
if __name__ == '__main__': 
    # generate the time(s) to event(s) 
    # AND count of events
    nhpp_event_times , events = gen_nhpp() 
    # Make plots
    fig , ax = plt.subplots()
    # step graph
    ax.step(nhpp_event_times , events , 
            label = f'$\lambda(t) = (1.01)^t$') 
    # Integrating the rate function to get E[N(t)]
    x = np.arange(140)
    y = 1.01**x/np.log(1.01) - \
    (1.01)**0/np.log(1.01)
    ax.plot(x , y , 
    	label = r'E[N(t)] = $\frac{1.01^t}\
            {ln(1.01)}$ - 100.499')
    # Auxiliaries
    ax.set_xlabel(r'$t$')
    ax.set_ylabel(r'$N(t)$')
    ax.set_title('Non-Homogeneous \
                 Poisson process')
    ax.legend(loc='best')
    plt.show()

\end{verbatim}

% The \nocite command causes all entries in a bibliography to be printed out
% whether or not they are actually referenced in the text. This is appropriate
% for the sample file to show the different styles of references, but authors
% most likely will not want to use it.
\nocite{*}

\bibliography{apssamp}% Produces the bibliography via BibTeX.

\end{document}
%
% ****** End of file apssamp.tex ******
