% ****** Start of file apssamp.tex ******
%
%   This file is part of the APS files in the REVTeX 4.2 distribution.
%   Version 4.2a of REVTeX, December 2014
%
%   Copyright (c) 2014 The American Physical Society.
%
%   See the REVTeX 4 README file for restrictions and more information.
%
% TeX'ing this file requires that you have AMS-LaTeX 2.0 installed
% as well as the rest of the prerequisites for REVTeX 4.2
%
% See the REVTeX 4 README file
% It also requires running BibTeX. The commands are as follows:
%
%  1)  latex apssamp.tex
%  2)  bibtex apssamp
%  3)  latex apssamp.tex
%  4)  latex apssamp.tex
%
\documentclass[%
 reprint,
 amsmath,amssymb,
 aps,
]{revtex4-2}

\usepackage{graphicx}% Include figure files
\usepackage{dcolumn}% Align table columns on decimal point
\usepackage{bm}% bold math
\usepackage{amsthm}%theorems and stuff
\usepackage[ruled,lined]{algorithm2e}
\usepackage{svg}
\usepackage{float}
\usepackage{hyperref}
\usepackage{enumitem}
\hypersetup{
    colorlinks=true,
    linkcolor=blue,
    filecolor=magenta,      
    urlcolor=cyan,
    pdftitle={Overleaf Example},
    pdfpagemode=FullScreen,
    }
\setlength\parindent{0pt} % zero indent
\theoremstyle{definition}
\newtheorem{definition}{Definition}[section]

%\usepackage[showframe,%Uncomment any one of the following lines to test 
%%scale=0.7, marginratio={1:1, 2:3}, ignoreall,% default settings
%%text={7in,10in},centering,
%%margin=1.5in,
%%total={6.5in,8.75in}, top=1.2in, left=0.9in, includefoot,
%%height=10in,a5paper,hmargin={3cm,0.8in},
%]{geometry}

\begin{document}

\preprint{APS/123-QED}

\title{ECMM450 Stochastic Processes\\Simulation of Non-Homogeneous Poisson Processes}% Force line breaks with \\
%\thanks{A footnote to the article title}%

\author{70054986}
 \affiliation{%
 Department of Computer Science, University of Exeter
}%


\date{April, 2023}% It is always \today, today,
             %  but any date may be explicitly specified

\begin{abstract}
This project is on Non-Homogeneous Poisson processes and how to simulate them. The thinning algorithm of Lewis and Shedler (1979) for simulating these processes will be discussed.

\end{abstract}

\maketitle

\section{\label{sec:level1}Introduction}

Homogeneous Poisson processes (HPP) are the natural representation of events occuring in time. As they follow a fixed rate of `event occurrence', they are unable to model processes with varying incidence rate, such as shocks before and after earthquakes \cite{vere-jones_1970}, fluctuations in neural data \cite{gabbiani_cox_2010}, events with time-of-day patterns or those showing long-term growth or decay.\\


This report will be structured according to the following points:
\begin{enumerate}
\item Explanation of a Non-Homogeneous Poisson Process (NHPP) accompanied by clear mathematical definition;
\item Review of the thinning algorithm of Lewis and Shedler, 1979 \cite{lewis_shedler_1979} for simulating NHPP, with brief description of the algorithm, and its main benefits compared to other approaches;
\item Simulation of occurrence of 1000 successive events ${t_1, t_2, . . . , t_{1000}}$ for a homogeneous Poisson process having a rate of 8 events per year, followed by visual representation of the same;

\item Determining whether the thinning algorithm can be used on the previous HPP data to simulate an NHPP with a rate function that increases smoothly from 1 event per year at $t = 0$ by 1\% per year, i.e. $\lambda(t) = (1.01)^t$;
\item Code to find occurence times for an NHPP having the rate function $\lambda(t) = (1.01)^t$ by using the thinning algorithm;
\item Comparison of the plots of HPP and NHPP to demonstrate differences between the two.

\end{enumerate}

\section{\label{q:1}Description and Mathematical Definition}
A non-homogeneous Poisson process can be thought of as a generalization of the homogeneous Poisson process. As opposed to its homogeneous counterpart where rate of occurrence of events is constant (denoted by $\lambda$), here the rate is a function of time, denoted by $\lambda(t)$. The process is specified either through its rate function $\lambda(t)$ (when it exists), or more generally through its expectation function $\Lambda(t) \equiv  E[N(t)]$ \cite{pasupathy}. $\Lambda(\cdot)$ is therefore, monotone, non-decreasing and right-continuous. \\

When the rate function $\lambda(t)$ exists, $\Lambda(t) = \int_{0}^{t}\lambda(y) dy$ \cite{pasupathy}. Thus, the number of occurrences in the interval $(0,T]$ follows Poisson distribution $Pois(\int_{0}^{T}\lambda(y)dy)$ \cite{chen}. More formally, allowing the rate parameter to vary with time results in the following definition.

\begin{definition}[Ross, 2006, p.339, Definition 5.4] \cite{Ross_2006} The counting process ${N(t), t \ge 0}$ is said to be a nonhomogeneous Poisson process with intensity function $\lambda(t) \ge 0, t \ge 0$, if
\begin{enumerate}
\item $N(0) = 0$
\item The process has independent increments
\item $P{N(t + h) - N(h) = 1} = \lambda(t)h + o(h)$
\item $P{N(t + h) - N(h) \ge 2} = o(h)$
\end{enumerate}
where $o(h)$ carries its usual meaning - a function $\phi(h)$, that satisfies the condition that $\frac{\phi(h)}{h}\to\infty$ as $h\to\infty$ \cite{billingsley_1995}.
\end{definition}

Time sampling an ordinary Poisson process (with constant rate $\lambda$) results in a Non-Homogeneous Poisson process. Given a Poisson process $\{N(t), t \ge 0\}$ with rate $\lambda$, if the event occurring at time $t$ is counted with probability $p(t)$, then the resulting point process $\{N_c(t) , t\ge 0\}$, is a Non-Homogeneous Poisson process \cite{Ross_2006}. It should be noted that unlike in HPP, the intervals between points in an NHPP are neither independent, nor identically distributed \cite{lewis_shedler_1979}.

\section{\label{q:2}Review of the Thinning Algorithm}

\subsubsection{Some Other Algorithms}
There exist other algorithms for simulating Poisson processes, a few of which are briefly described.
\begin{enumerate}
\item[a.] One is the time-scale transformation of a homogeneous Poisson process via inverse of the integrated rate function $\Lambda(x)$. Cinlar \cite{cinlar1975introduction} states that the random variables $T1, T2, . . .$ are event times corresponding to a nonhomogeneous Poisson process with expectation function $\Lambda(t)$ if and only if $\Lambda(T1),\Lambda(T2), . . .$ are the event times corresponding to a homogeneous Poisson process with rate one. This of course requires the inverse of the integrated rate function $\Lambda(t)$ to be computed, which may prove costly and inefficient.

\item[b.] Another method is to generate intervals between the points individually \cite{pasupathy}. This may seem more in line with the event-scheduling approach to simulation. It follows a similar inversion as the first method. Considering the $i$th inter-event time $X_i = T_{i+1} - T_i$, conditional on the first $i$ event times $T_1 = t_1, T_2 = t_2, . . . , T_i = t_i$, its cdf can be derived as $Fti = 1 - e^{-\Lambda(ti + x) + \Lambda(ti)}$. Then $X_{i+1} - X{i}$ can be generated as $X_{i+1} - X{i} = F^{-1}(U_i)$, where $U_i \sim Unif(0,1)$. As stated in the original paper \cite{lewis_shedler_1979}, this process involves not only the computation of the inverse cdf, but the distribution has different parameters for each interval.

\item[c.] Simulation of Non-Homogeneous Poisson variates can be done through by generating order statistics from a distribution with cdf $F(t) = \Lambda(t)/\Lambda(t_0)$ for $t\in [0,t_0]$ \cite{lewis_shedler_1976}. Formally, let $T_1 , T_2 , ...$ be time(s) to event for a Non-Homogeneous Poisson process with integrated rate function $\Lambda(t)$. Conditional on having observed $N(t_0) = n$ points in $(0,t_0]$, $T_k$ is the $k$th order statistic from a distribution with cdf $F(t) = \Lambda(t) / \Lambda(t_0)$. Ordering can prove to be computationally expensive.
\end{enumerate}

\subsubsection{The Thinning Algorithm}
To construct a Non-Homogeneous Poisson process $\{N(t) , t \ge 0 \}$, with rate parameter $\lambda(t)$, over the interval $(0,T]$, the algorithm starts with a Non-Homogeneous Poisson process $\{N^{*}(t) , t \ge 0 \}$, with rate parameter $\bar{\lambda}(t)$ that dominates the set $\lambda(t)$ for all $t \in (0,T]$, that is 
\begin{gather*}
\bar{\lambda}(t) \ge \lambda(t)\  \forall t \in (0,T]\\
\bar{\lambda}(t) = sup_{t\in(0,T]}\lambda(t)
\end{gather*}
Then, for all $t$, the point from the dominating NHPP $N^{*}(t)$, is retained with probability $\lambda(t)/\bar{\lambda}$. The remaining points form an NHPP with rate parameter $\lambda(t)$. It is noted that since points are deleted independently, the number of points in $\{N(t) : t \ge 0\}$ in any set of non-overlapping intervals are mutually independent.

\begin{algorithm}
\label{alg:1}
\caption{(Lewis and Shedler, 1979, p.7, Algorithm 1) Simulation of an Inhomogeneous Poisson Process with Bounded Intensity Function $\lambda(t)$, on $[0, T]$}
\textbf{Input}: $\lambda,T$

Initialize $n = m = 0$, $t_0 = s_0 = 0$, $\bar{\lambda}(t) = sup_{t\in(0,T]}\lambda(t)$;

\While{$s_m < T$}{
Generate $u\sim$\texttt{uniform(0,1)}\; 
Let $w = -ln (u) / \bar{\lambda}$\; %\rcomment{// so that $w\sim Expo(\bar{\lambda})$}
Set $s_{m+1} = s_m + w$\;
Generate $D\sim$ \texttt{uniform(0,1)}\;
\If{$D < \lambda(s_{m+1})/\bar{\lambda}$}{
   $t_{n+1} = s_{m+1}$\;
   $n = n+1$\;
   }
$m = m + 1$
}
\eIf{$t_n \le T$}{
\Return $\{t_k\}_{k=1,2,...,n}$
}{
\Return $\{t_k\}_{k=1,2,...,n-1}$
}
\end{algorithm}

\subsubsection{Benefits Over Other Algorithms}
Each method has drawbacks with respect to computational efficiency. In its simplest implementation, the thinning method ``\textit{obviates the need for numerical integration of the rate function, for ordering of points, and for generation of Poisson variates}'' \cite{lewis_shedler_1979}. 

\section{\label{q:3}Simulation of a Homogeneous Poisson process}

\begin{algorithm}
\label{alg:2}
\caption{Simulation of a Homogeneous Poisson Process with Rate $\lambda$, on $[0, T]$. Slight variation of Algorithm 1 in Chen, 2016 \cite{chen}}
\textbf{Input}: $\lambda,N$

Initialize $n_0 = 0 ,\  t_0 = 0$\;
\While{$True$}{
Generate $u\sim$\texttt{uniform(0,1)}\; 
Let $w = -ln (u) / \bar{\lambda}$\; %\rcomment{// so that $w\sim Expo(\bar{\lambda})$}
Set $t_{n+1} = t_n + w$\;
\eIf{$n+1 > N$}{
\Return $\{t_k\}_{k=1,2,...,n}$
}{
Set $n = n + 1 $\;
}
}
\end{algorithm}

Uniform random numbers are used to generate Poisson variates by using algorithm \ref{alg:2}, that can be found in \cite{chen}. Full implementation in Python can be found in the Appendix. FIG \ref{fig:hpp} shows the result of simulating a homogeneous Poisson process.

\begin{figure}[H]
\centering
\includegraphics[scale=0.5]{HPP.png}
\caption{\label{fig:hpp}Simulation of the time of occurrence of 1000 successive events for a homogeneous Poisson process having a rate of 8 events per year. $N(t)$ denotes the number of events occurring in the time period $[0,t]$.}
\end{figure}

The graph appears relatively linear. This would be expected as the rate of occurrence of events remains constant over any interval of time $(a,b)$ and hence the slope corresponding to the number of events over time (i.e. rate) also remains the same. A comparison against $E[N(t)] = \lambda t$ has been plotted in FIG \ref{fig:hppent} to demonstrate the linearity of the HPP.

\begin{figure}[H]
\centering
\includegraphics[scale=0.5]{HPP_ENt.png}
\caption{\label{fig:hppent}Expected number of events follows a straight line with slope, $\lambda$ = 8. $N(t)$ denotes the number of events occurring in the time period $[0,t]$.}
\end{figure}

\section{\label{q:4}NHPP with a smoothly increasing rate function}

An NHPP is considered with a rate function that increases smoothly from 1 event per year at $t = 0$ by $1\%$ per year, i.e. $\lambda(t) = (1.01)^t$. Looking at the previous homogeneous Poisson process from \ref{q:3}, it can be seen that the maximum time taken to accumulate 1000 events is $\sim140$. FIG \ref{fig:hpp_many_sim} shows this. Therefore, the maximum rate for the NHPP would be $(1.01)^{140} = 4.027$. The rate for the previous homogeneous Poisson process was 8. Hence, this is within the bounds of the original process. Thinning can be applied to simulate the NHPP from the previous HPP data.

\begin{figure}
\centering
\includegraphics[scale=0.5]{HPP_many_sim.png}
\caption{\label{fig:hpp_many_sim}Time in years to accumulate 1000 events at 8 per year. 10,000 iterations performed to generate histogram. Y-axis represents density.}
\end{figure}

\section{\label{q:5}Performing the Thinning Algorithm with data from Homogeneous Poisson Process}
Using the thinning algorithm, the data points from the previous homogeneous process are time-sampled to construct a Non-Homogeneous process. Code can be found in Appendix \ref{app:nhppcode}.

\section{\label{q:6}Graphical Comparison of HPP and NHPP}
Fig \ref{fig:nhpp} showcases the cumulative events $N(t)$ against time $t$. It is observed that, as opposed to homogeneous Poisson process, the line is not quite linear - the slope increases as time progresses. This can be expected as the rate parameter $\lambda(t) = (1.01)^t$ also increases with time. Due to deletion of points from the original homogeneous process, it is also noted that the total number of points has decreased from 1000 to just over 300.

\begin{figure}
\centering
\includegraphics[scale=0.5]{NHPP.png}
\caption{\label{fig:nhpp}Accumulated events following an NHPP with rate parameter $\lambda(t) = 1.01^t$. $N(t)$ represents cumulative events. $t$ denotes time.}
\end{figure}

By integrating the rate function, we get 
\begin{gather*}
\int_{0}^{t}\lambda(t)dt = \int_{0}^{t}1.01^tdt\\
=1.01^t/ln(1.01) + C
\end{gather*}

Noting the boundary condition that at $t=0$, $E[N(t)]=0$, we have $C = \frac{1.01^0}{ln(1.01)} = 100.499$.\\

\begin{figure}[H]
\centering
\includegraphics[scale=0.5]{NHPP_ENt.png}
\caption{\label{fig:nhppent}$E[N(t)]$, denoted by $\frac{1.01^0}{ln(1.01)} - 100.499$ superimposed on the NHPP. $N(t)$ represents cumulative events. $t$ denotes time.}
\end{figure}

Thus,  $E[N(t)] = \frac{1.01^t}{ln(1.01)} - 100.499$. Plotting this onto the earlier graph, FIG \ref{fig:nhppent} is obtained. It can be seen that the orange line closely follows the blue stepped graph.

\appendix

\section{\label{app:hppcode}Code for HPP}
Below is the code for section \ref{q:3}. It simulates a homogeneous Poisson process with rate parameter $\lambda=8$ per year. Time to 1000 events is simulated and the graph generated from this code is FIG \ref{fig:hpp}. .py files can be found \href{https://github.com/sourasen1011/Stochastic_Processes/blob/main/CA3/project_helper/HPP.py}{here}.
\begin{verbatim}
# Imports
import numpy as np
import matplotlib.pyplot as plt
# define hpp function
def gen_hpp(lmbda , N):
    '''
    param lmbda: rate parameter
    N: Number of events
    '''
    # inits
    t = [0]
    # begin loop
    while True:
        # generate uniform r.v. ~ Unif[0,1]
        u = np.random.uniform(0,1)
        # generate w ~ Exponential(lambda)
        w = - np.log(u)/lmbda
        t.append(t[-1] + w)
        # exit condition
        if len(t) > N:
            # get time to event & count the
            # number of events
            return t , np.arange(len(t))

if __name__ == '__main__': # main namespace
    l , N = 8 , 1000
    # generate the time(s) to event(s)
    # AND count of events
    hpp_event_times , events = \
    gen_hpp(lmbda = l , N = N)
    print(events , hpp_event_times) # debug
    # Make plots
    fig , ax = plt.subplots()
    # step graph
    ax.step(hpp_event_times , events ,
    label = f"$\lambda$ = {l}" , lw = 0.5)
    # Integrating the rate function to 
    # get E[N(t)]
    x = np.arange(120)
    y = l*x
    # Plot
    ax.set_xlabel(r'$t$')
    ax.set_ylabel(r'$N(t)$')
    ax.set_title('Homogeneous Poisson process')
    ax.legend(loc='best')
    plt.show()
\end{verbatim}

\section{\label{app:nhppcode}Code for NHPP}
Below is the code for section \ref{q:5}. It simulates a Non-Homogeneous Poisson process with rate parameter $\lambda(t)=1.01^t$ that smoothly increases from 1 event per year. The time boundary is kept at 140 years as this is roughly the longest time it took the HPP to reach 1000 events. The graphs generated from this code are FIG \ref{fig:nhpp} and FIG \ref{fig:nhppent}. In the latter, $E[N(t)]$ is shown along with the realizations from the NHPP to that both the lines adhere to each other. .py files can be found \href{https://github.com/sourasen1011/Stochastic_Processes/blob/main/CA3/project_helper/NHPP.py}{here}.
\begin{verbatim}
# Imports
import numpy as np
import matplotlib.pyplot as plt

# define hpp function
def gen_nhpp(lmbda_bar = 8 ,  T = 140):
    '''
    param lmbda_bar: rate parameter that 
    dominates the rate param of the NHPP
    T : maximum time
    '''
    # inits
    s = [0]
    t = [0]

    # begin loop
    while s[-1] < T:
        # generate uniform r.v. ~ Unif[0,1]
        u = np.random.uniform(0,1)
        # generate w ~ Exponential(lambda) 
        w = - np.log(u)/lmbda_bar 
        s.append(s[-1] + w)
        # geenrate D ~ Unif[0,1]
        D = np.random.uniform(0,1)
        # acceptance criterion
        if D < (1.01)**s[-1] / lmbda_bar:
            t.append(s[-1])
        
        if t[-1] > T:
            num_events = np.arange(len(t[:-1]))
            print(f'Breakpoint 1: the number of \
            events is {num_events[-1]}, and the \
            time taken to reach them is {t[:-1][-1]}')
            # get time to event & count the 
            # number of events
            return t[:-1] , num_events
    else:
        num_events = np.arange(len(t))
        print(f'Breakpoint 2: the number of \
        events is {num_events[-1]}, and the \
        time taken to reach them is {t[-1]}')
        # get time to event & count the number 
        # of events
        return t , num_events

# main namespace
if __name__ == '__main__': 
    # generate the time(s) to event(s) 
    # AND count of events
    nhpp_event_times , events = gen_nhpp() 
    # Make plots
    fig , ax = plt.subplots()
    # step graph
    ax.step(nhpp_event_times , events , 
            label = f'$\lambda(t) = (1.01)^t$') 
    # Integrating the rate function to get E[N(t)]
    x = np.arange(140)
    y = 1.01**x/np.log(1.01) - \
    (1.01)**0/np.log(1.01)
    ax.plot(x , y , 
    	label = r'E[N(t)] = $\frac{1.01^t}\
            {ln(1.01)}$ - 100.499')
    # Auxiliaries
    ax.set_xlabel(r'$t$')
    ax.set_ylabel(r'$N(t)$')
    ax.set_title('Non-Homogeneous \
                 Poisson process')
    ax.legend(loc='best')
    plt.show()

\end{verbatim}

\nocite{*}

\bibliography{apssamp}% Produces the bibliography via BibTeX.

\end{document}
%
% ****** End of file apssamp.tex ******
